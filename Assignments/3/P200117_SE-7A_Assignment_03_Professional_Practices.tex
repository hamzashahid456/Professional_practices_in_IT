\documentclass{article}
\usepackage{graphicx} % For inserting images

\title{Assignment 03: Professional Practices}
\author{P200117 Hamza Shahid SE-7A}
\date{November 25, 2023}

\begin{document}

\maketitle

\section{Introduction}
Gucci is a multinational company with over 270 directly operated stores worldwide, serving customers of elite goods and generating billions of dollars in revenue per year. It has an iconic, even noble, luxury brand image in the Greater China region, where its revenue increased by 35.6\% in the first half of the year 2011.

\section{Case description}
On 8 October 2011, an open letter—<A Public Letter to the Top Management of Gucci from Former Employees who resigned collectively> was spread on the Internet. This letter was written by five former employees of the Gucci Shenzhen Flagship Store. In the letter, they alleged that employees caught an occupational disease, that there was one miscarriage attributable to excessive working hours and that there was no compensation for these hardships. Moreover, they stated that there were excessive restrictions on employees’ behavior, including the need to obtain permission before getting a drink or a snack, and strict limitations on toilet time. They stated that, while the restrictions were applied strictly to all frontline employees, including one who was pregnant, they were not applied to the managers.

The letter also claimed that the employees had to pay compensation for any product that was stolen or went missing, even though these luxury products had already been insured. They also criticized Gucci’s goods exchange policies which appeared to be arbitrary and dependent on the manager’s mood. All in all, they accused Gucci of lacking systematic and humane management and complained that their rights and dignity were being violated.

\section{Stakeholder Network}
Dispatch is a labor management model which separates recruitment from employment. Relationships under the dispatch system are portrayed in Fig. 1. 

\begin{figure}
    \centering
    \includegraphics[width=0.5\linewidth]{Screenshot from 2023-11-25 21-10-48.png}
    \caption{Relationships under the dispatch system}
    \label{fig:enter-label}
\end{figure}

The employee leasing companies have labor contracts with the workers, and they send workers to other companies in which these workers actually work. The labor contract relationship exists between the employee leasing companies and the dispatched workers, but the actual working relationship is between the workers and the companies in which they work.

In this form of employment, the company which actually “use” these workers is only responsible for paying wages, while other aspects, including social security and dismissal compensation are passed on to the employee leasing company. The labor dispatch arrangement serves to reduce the user companies’ costs and contractual responsibilities for the employee. They can incur lower training costs and are not required to make social security arrangements. Because of these features, this employment model is widely used in China. The Gucci stores in Shenzhen actually adopted an even more complex dispatch system, involving at least three employee leasing companies that were located in Shanghai.

\section{Legal Consideration}
One legal consideration is that, although the labor dispatch system has been officially adopted as way of arranging temporary employment only, Gucci used the system to employ people for durations of more than 2 years. Another is that many of the Gucci store employees are female and that pregnant employees legally enjoy special labor protection. According to the “Labor Contract Law,” female workers during their pregnancy should not participate in the state’s third-grade physical intensive work. Such work is deemed not suitable for female workers; for female workers who are more than 7 months pregnant, there should be no overtime work, and they should not be required to join night shifts. Furthermore, it is a legal requirement that sufficient rest periods should be arranged for such employees.



\end{document}

