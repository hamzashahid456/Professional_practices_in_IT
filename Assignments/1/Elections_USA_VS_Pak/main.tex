\documentclass{article}

\title{Law and order system of USA and Pakistan}
\author{P200117 Hamza Shahid}
\date{September 3, 2023}

\begin{document}

\maketitle

\section{Elections}
\subsection{U.S.A:}
The United States operates under a federal presidential system, with a separation of powers among the executive, legislative, and judicial branches of government. Elections held at various level of government including federal, state and local levels.
The two main types of elections are as follow:

% types of elections
\begin{itemize}
  \item \textbf{Presidential Elections:} 
  It held every four year to elect president and vise president
  \item \textbf{Midterm Elections:}
  It Held every two years to elect members of the U.S. Congress (House of Representatives and Senate), as well as state and local officials.
\end{itemize}
% How presidential elections done
In presidential elections, the U.S. uses an Electoral College system where citizens vote for a slate of electors who then cast their votes for the President. The candidate who receives a majority of electoral votes (270 out of 538) wins the presidency.
\newline
% Political parties 
The U.S. primarily has a two-party system dominated by the Democratic Party and the Republican Party, although there are other minor parties as well.
\newline
% Compaigns
Campaigns in the U.S. involve extensive fundraising, advertising, and grassroots organizing. Candidates often engage in debates, town hall meetings, and other events to connect with voters.

\subsection{Pakistan}
Pakistan operates under a parliamentary system, with a President as the ceremonial head of state and a Prime Minister as the head of government. The National Assembly (lower house) and Senate (upper house) are the main legislative bodies.
% Elections
\newline
General elections in Pakistan are held to elect members of the National Assembly, Provincial Assemblies, and other local bodies. Provincial assemblies also elect senators.
% Prportional Representation
\newline
Pakistan uses a mixed electoral system that includes both first-past-the-post (FPTP) and proportional representation. Some seats in the National and Provincial Assemblies are allocated to political parties based on the proportion of votes they receive.
% Politiclal parties
\newline
Pakistan has a multi-party system with several political parties representing various interests and ideologies. Like Pakistan Tehreek-e-Insaf (PTI) and Pakistan Peoples Party (PPP).
% Compaigns
\newline
Election campaigns in Pakistan involve rallies, public meetings, and media advertising. There is a strong presence of political dynasties, and candidates often rely on their family's political legacy.
\newline
\subsection{Differences}
\textbf{Presidential vs. Parliamentary System:} The U.S. has a presidential system, while Pakistan has a parliamentary system.
\newline
\textbf{Electoral College vs Direct Elections:} The U.S. uses an Electoral College system in presidential elections, whereas Pakistan uses a direct voting system for the National Assembly and Provincial Assemblies.
\newline
\textbf{Separation of Powers:} The U.S. has a clear separation of powers between the executive, legislative, and judicial branches, while Pakistan's system combines elements of both parliamentary and presidential systems.
\newline
\textbf{Proportional Representation:} Pakistan incorporates proportional representation in its electoral system, which is less common in the U.S.
\newline
\textbf{Campaign Dynamics:} Election campaigns in the U.S. are often highly competitive and heavily funded, while Pakistan's political landscape includes a mix of established political families and diverse parties.

% Law making and check and balance
\section {Law Making and Enforcement}
% In USA
\subsection{U.S.A:}
\textbf{Proposal:} A bill can be introduced in either the House of Representatives or the Senate. It can be proposed by members of Congress, the President, or other entities.
\newline
\textbf{Committees:} Bills are referred to relevant committees for review, where they are debated, amended, and voted upon.
\newline
\textbf{Floor Debate:} If a bill is approved by committee, it goes to the floor of the House or Senate for debate and voting.
\newline
\textbf{Conference Committee:} If the House and Senate pass different versions of a bill, a conference committee is formed to reconcile the differences.
\newline
\textbf{Presidential Action:} If both chambers of Congress agree on a bill, it is sent to the President for approval or veto.

\subsubsection{Check and Balance:}
\textbf{Separation of Powers:} The U.S. Constitution separates powers among the executive (President), legislative (Congress), and judicial (courts) branches of government.
\newline
\textbf{Veto Power:} The President has the power to veto legislation passed by Congress, which can be overridden by a two-thirds majority vote in both the House and Senate.
\newline
\textbf{Judicial Review:} The federal courts, including the Supreme Court, have the authority to review laws for constitutionality and can declare them invalid if they violate the Constitution.
\newline
\textbf{Congressional Oversight:} Congress exercises oversight over the executive branch through hearings, investigations, and budgetary control.

\subsection{Pakistan:}
\textbf{Proposal:} Bills can be introduced in either the National Assembly or the Senate. Members of Parliament, the President, or the Cabinet can propose bills.
\newline 
\textbf{Committees:} Bills are referred to parliamentary committees for review and recommendations.
\newline
\textbf{Debate and Voting:} Bills are debated and voted upon in the National Assembly and the Senate. They require a simple majority to pass.
\newline
\textbf{Presidential Action:} If both houses pass a bill, it is sent to the President for approval. The President's approval is typically ceremonial but can be significant in certain cases.

\subsubsection{Check and Balance:}
\textbf{Parliamentary System:} Pakistan operates under a parliamentary system where the executive (Prime Minister) is a member of the legislature (National Assembly).
\newline
\textbf{Prime Minister's Control:} The Prime Minister and Cabinet are drawn from the majority party in the National Assembly, ensuring legislative and executive alignment.
\newline
\textbf{No Presidential Veto:} Unlike the U.S. President, the Pakistani President does not have veto power over legislation.
\newline
\textbf{Judicial Review:} The judiciary in Pakistan, including the Supreme Court, has the authority to review laws and government actions for conformity with the Constitution.

\subsection{Differences}
\textbf{System of Government:} The United States has a presidential system with a clear separation of powers, while Pakistan has a parliamentary system where the executive is part of the legislature.
\newline
\textbf{Presidential Veto:} The U.S. President has a significant veto power over legislation, which is not present in Pakistan.
\newline
\textbf{Bicameralism:} Both countries have a bicameral legislature, but the roles and powers of the two houses differ. In the U.S., they are equal, while in Pakistan, the National Assembly is more powerful.
\newline
\textbf{Judicial Review:} Both countries have a judicial branch with the power of judicial review, but the extent and history of judicial activism vary.


\end{document}
